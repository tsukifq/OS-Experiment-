% !Mode:: "TeX:UTF-8"

%%% 此部分需要自行填写: 中文摘要及关键词 

%%% 郑重声明部分无需改动

%%%---- 郑重声明 (无需改动)------------------------------------%
\newpage
\thispagestyle{empty}
\vspace*{20pt}
\begin{center}{\ziju{0.8}\pmb{\songti\zihao{2} 郑重声明}}\end{center}
\par\vspace*{30pt}
\renewcommand{\baselinestretch}{2}

{\zihao{4}%

本人呈交的设计报告,是在指导老师的指导下,独立进行实验工作所取得的成果,
所有数据、图片资料真实可靠。 尽我所知,除文中已经注明引用的内容外,
本设计报告不包含他人享有著作权的内容。
对本设计报告做出贡献的其他个人和集体,
均已在文中以明确的方式标明。本设计报告的知识产权归属于培养单位。\\[2cm]

\hspace*{1cm}本人签名: \underline{秦槿\hspace{2cm}}
\hspace{2.0cm}
日期: \underline{2023.8.8\hspace{1cm}}}
%------------------------------------------------------------------------------
\baselineskip=23pt  % 正文行距为 23 磅
%------------------------------------------------------------------------------





%%======摘要===========================%
\begin{cnabstract}
\thispagestyle{empty}

操作系统课程设计旨在通过模拟实践,引导学生一步步设计一个操作系统。逐步构建操作系统加深了对引导盘、保护模式、操作系统内核、进程、键盘IO以及处理器调度等概念的理解,同时实现了对操作系统设备管理的掌握。此实验帮助学生更深刻地领会课程理论,并直观认识实际操作系统内核。\par
实验设计遵循操作系统关键理论,包括保护模式、进程、内核、IO处理。涵盖NASM汇编语言和C语言等内容。实验步骤包括搭建Linux虚拟机和Bochs仿真环境,在自己搭建的环境中调试运行《ORANGE'S:一个操作系统的实现》的前七章代码,并在此基础上自主实现附加实验——实验9的要求。\par
作为计算机系统中至关重要的组成部分,操作系统承担着管理和控制计算机系统软硬件资源的重要任务,其设计过程需要进行繁多且详尽的考量。在实验过程中,我深深感受到操作系统的每一个组成部分都紧密地相互关联、相互影响。从保护模式到内核雏形,再到进程管理和设备管理,每个环节协同工作才能使得一个操作系统运转良好,为用户提供稳定可靠的计算环境。通过编写和调试代码,我逐渐深入地理解了操作系统的运行机制,也学会了如何进行处理器进程调度,如何管理设备输入,以及如何使用汇编语言和 C 语言来实现操作系统的各个功能模块。\par
这些实践经验不仅深化了我的理论知识,也增强了我解决实际问题的动手能力。在未来的学习和工作中,我将继续保持深入研究操作系统领域的学习热情,不断提升自己的知识水平和技能,并期待能够参与更复杂、更庞大的操作系统项目,为构建高效、安全的计算环境做出贡献。


\end{cnabstract}
\par
\vspace*{2em}


%%%%--  关键词 -----------------------------------------%%%%%%%%
%%%%-- 注意: 每个关键词之间用“;”分开,最后一个关键词不打标点符号
\cnkeywords{操作系统复现;NASM汇编语言;Linux;bochs;}



