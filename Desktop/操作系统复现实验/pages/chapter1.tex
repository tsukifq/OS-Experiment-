\chapter{操作系统复现内容总览}

\section{实验环境的搭建}
实验内容:在 Windows 操作系统下安装虚拟机软件(如 Virtualbox,VMware等),然后在虚拟机上安装 Ubuntu 操作系统。安装 Bochs 模拟器 Bochs 模拟 X86硬件平台,并利用其自带 bochsdbg 调试器调试操作系统设计代码。\par
基本步骤:安装虚拟机;配置 Linux 环境;安装 Bochs 模拟器;安装 nasm、gcc、gnu 等工具;操作系统引导盘的制作;运行引导扇区。


\section{动手写一个最小的操作系统}
实验内容:通过编译一段最基本的 asm 代码来初次体验操作系统的设计以及了解 NASM 编译的使用方法、dd 命令写入磁盘的方法以及 Bochs 的使用方法。\par
基本步骤:NASM编译;dd 命令写入磁盘,软盘读写工具将文件写到空白软盘的第一个扇区;调用子程序显示字符串,并无限循环;bximage 命令创建一个空白映像文件,再使用dd 命令将.bin 文件写入;撰写bochsrc文件,配置 bochs 基本信息,使用.bin 文件从指定软盘启动;启动后进行调试。


\section{实现保护模式}
实验内容:认识保护模式,实现从实模式到保护模式的转换,GDT 描述符;实现实模式大于 1MB 内存的寻址能力,并接着上一次实验,从保护模式返回到实模式,重新设置各个段寄存器的值;LDT 描述符;学会使用挂载指令和运行程序。\par
实验步骤:借助 DOS 扩展程序范围,安装FreeDos,并将两个软盘映像都写进bochs配置文件;学习数据结构 GDT及其描述符;学习数据结构 LDT,关注其区别于GDT的局部特性;把.asm 文件编译成.com 文件,使用挂载指令将该文件复制到软盘映像文件里;了解保护模式如何获取大于 1MB 寻址能力;认识实模式和保护模式以及互相跳转的方式。

\section{切换到保护模式}
实验内容:引导扇区突破 512 个字节的限制,将工作分给 loader;加载 loader 进入内存并运行;将控制权交给 loader。\par
基本步骤:突破 512 字节的限制:交给 loader 来完成;认识 FAT12,遍历根目录区所有的扇区,将每一个扇区加载入内存,从中寻找 Loader.bin;使用 jmp 指令跳到内存中 loader.bin 的开始处;向 loader交出控制权。

\section{内核雏形}
实验内容:在 Linux 下用汇编写 Hello, World!;进一步,汇编和 C 同时使用;从 loader 到 kernel 内核,把 kernel 内核加载到内存;将控制权交给 kernel 内核;跳入保护模式,并显示内存的使用情况。\par
基本步骤:linux 汇编下的 hello world:helloworld 编译成 elf 格式;./hello 运行;汇编和 C 同时使用:通过global 导出,extern 声明;./hello 运行;从 loader 到内核:用 loader 把 kernel 加载到内存;进入保护模式;控制权交给内核:把 Kernel.bin 加载到内存。

\section{进程及进程调度}
实验内容:进程切换;丰富中断处理程序,比如让时钟中断处理可以不停地发生而不是只发生一次,进程状态的保存与恢复,进程调度,解决中断重入问题。\par
基本步骤:学习最简单的进程:进程表、进程栈、内核栈;特权级变换;寄存器值的压栈和恢复;准备进程体;初始化相应描述符;准备进程表;完成特权级别的跳转;丰富中断处理:设置 EOI、现场的保护和恢复、中断重入;多进程处理:读取不同的任务地址入口、堆栈栈顶和进程名;进程切换;系统调用;进程调度。

\section{输入/输出系统}
实验内容:实现简单的 I/O,从键盘输入字符的中断开始;获取并打印扫描码;创建对应打印扫描码解析数组,打印对应字符。\par
基本步骤:添加中断指令,多次输入;打印 make code 和 break mode;建立键盘输入缓冲区分析字符;处理 shift、alt 和 ctrl相关组合键;

\section{操作系统进阶}
实验内容:自定义一个系统调用,能够统计一个进程在执行的过程中被调度的次数。编写一个简单的用户程序,调用该自定义的系统调用,从而将进程及其调度的次数输出在屏幕上。并在实现了三个进程的优先级调度的基础上,将三个进程的循环次数从无限循环修改为有限次数,当三个进程执行完成时,计算三个进程在优先级调度算法下的周转时间、等待时间以及该系统的平均周转时间、平均等待时间和吞吐量。也可以添加新的进程,然后计算该系统的平均周转时间、平均等待时间和吞吐量。\par
基本步骤:从系统调用开始:定义系统调用函数体;定义函数声明;添加 sys$\_$call$\_$table 成员;增加NR$\_$SYS$\_$CALL的值;返回输出用户进程的函数调用次数;初始化进程表同时定义进程优先级和调用次数;设计进程调度算法;返回输出系统的平均周转时间、平均等待时间和吞吐量、
