\chapter{实验心得体会}\label{cha:latex-brief-intro}

本次操作系统复现实验中,我参考《一个操作系统的实现》\cite{于渊2009orange}这本指导书,将理论知识与实践相结合,复现了一个基本的操作系统。从搭建虚拟机工作环境开始,一步步地实现保护模式和实模式的转换,切换到保护模式之后实加载 Loader 进入内存并将控制权交给 loader,随后将 loader 替换成 kernel 内核,实现进程切换、保存与恢复、调度,并解决可能会出现的中断重入问题。最后实现 I/O 系统,一个由我们自己设计的简单的操作系统就此完成。在这次实验过程中,我通过每一次实验之间代码的不同以及最后呈现的效果的区别,对不同代码对应的操作系统的基本结构、原理和功能有了更清晰明确的理解,也对NASM汇编语言有了初步的认识。\par
这次实验首先深化了我对操作系统的理解。我对操作系统的多进程管理、输入输出处理等方面有了更加深刻和直观的认识。通过亲自实践,我认识到了操作系统是如何协调和管理多个进程的执行,如何处理输入输出请求等。
除此之外,我还增加了对 CPU 硬件的了解。实现一个操作系统不可避免地需要与硬件进行交互,而这次实验让我对一些关键芯片有了初步的了解。其中,我对中断控制芯片 8259A、时钟控制芯片 8253 以及键盘控制芯片 8042 有了初步的认识。了解这些芯片的功能和工作原理,让我意识到操作系统是如何与硬件进行通信和控制的。
最后,在这次实验中,我还初步接触了NASM汇编语言,发现与此前学过的x86汇编语言较为类似,在此基础上通过阅读源码对比并实践运行,我逐渐掌握了NASM汇编语言的基本使用方式,以及编写汇编指令来实现特定的功能,与C语言进行混合编译。\par
由于之前的一些科研工作涉及到过Linux系统,这次实验的环境配置部分没有给我带来过多困扰,但在此前我更多地是了解学习Linux系统的相关指令。在这次实验的操作过程中,我开始从原理上深化理解这些指令的意义,这对我的实操能力带来了很大提升。\par
同时,我非常感谢老师的指导和帮助以及提供的参考资料\cite{于渊2009orange}\cite{mit},没有这些内容作为参考,我无法完成这个实验。通过这次实验,我不仅掌握了操作系统的基本原理和实践技能,还培养了解决问题的能力和动手实践的精神。这对我的职业发展和学术研究都具有重要意义。我将继续深入学习操作系统的知识,并将其应用到未来的学习和研究中。

